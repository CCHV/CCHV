\section{Discussion}

We studied in this simulation work the performance of the CLaRyS Compton camera prototype and its possible implementation as prompt gamma detector for ion beam therapy monitoring. The proposed analysis is focused on three main points: absolute gamma detection efficiency, true and random coincidence rate and camera precision in the identification of the prompt gamma emission profile fall-off.

The absolute gamma detection efficiency has been measured with the detector exposed to gamma sources at six different energies, and the efficiency variation as a function of the source position has been reported. The whole study has been performed with a fixed camera geometry, but different geometrical configuration can change the obtained results. In~\cite{Fontana_APPB} we tested the same prototype with two absorber configurations, showing an absolute efficiency reduction of a factor approximately two when reducing the absorber from a $8\times6$ block matrix to a $3\times3$ one. To this efficiency reduction corresponds an increased spatial resolution, probably given by the selection of gamma scattered with small angles. This hypothesis is confirmed by the efficiency variation generated by the applied energy thresholds. Indeed, for low energies, below 2~MeV, the increased efficiency in the central section of the camera active surface shown in figure~\ref{fig::efficiency_study}(a) is linked to the increased relative number of photons approaching the camera with small angles. These photons are more likely undergoing Compton scattering with a reduced energy deposition, which is recorded by an ideal detector and rejected by the energy threshold. The effect is all the more important as the primary gamma energy is limited, creating the peculiar energy dependence of the efficiency reduction observed in the results in figure~\ref{fig::efficiency_study}(b). Therefore, in case the compactness of the detector must be privileged, the efficiency reduction is not dramatic for the Bragg peak monitoring purpose. Moreover, such a compact configuration allows for the implementation of several detector heads, which can be set at different angles and provide additional spatial information. In case the highest possible efficiency is required, the distance between scatterer and absorber can be modified, knowing that for reduced distances the efficiency is increased because a minimal amount of scattered photons escapes the absorber field of view. On the other hand, a reduction of the inter-detector distance is not recommended for accurate TOF measurements (for fixed target-scatterer distance), given the uncertainty added by the detectors time resolutions. In addition to this, also the spatial resolution is affected by a reduction of the scatterer-absorber distance, given the increased amount of photons scattered with large Compton angles included in the collected events. Indeed, large angle scatterings tend to degrade the spatial resolution.   
Regarding the dependence of the efficiency on the Compton camera position, an accurate setup with respect to the expected Bragg peak position appears mandatory for the detection optimization.

A further study showed how this effect is mainly due to the cut applied on the scatterer detector. 
%The results emerged from the analysis of the rate of true and random coincidences with respect to the beam intensity show how at a realistic clinical intensity the signal-over-noise ratio is very low (less than 0.1) for both raw and reconstructed events and for proton and carbon beams. 
The study of the signal to noise ratio as a function of beam intensity was performed (see figure \ref{fig:coincidences}). The results show that, at clinical intensities, this ratio is very low. The TOF selection is effective in the case of carbon beams, where a significant proton and neutron contamination is expected. If we consider the case of proton beams, the amount of random and true coincidences is comparable at an intensity of about 1 proton per bunch, so that a clinical intensity reduction should be necessary to fit this configuration. Even if a possible image reconstruction is not excluded by the low signal-over-noise ratio detected at realistic beam intensity, given the fact that the random coincidence are distributed in an homogeneous way in the reconstructed volume, an intensity reduction can be an option in order to obtain more significant data sets. It must be noticed that the need for Bragg peak position online check is all the more necessary for distal spots, which are in general the first to be treated, so that a beam intensity reduction at the beginning of the treatment can be foreseen in case an accurate monitoring is strictly needed. This would not affect the treatment delivery, nor the planned patient rate in the clinical work-flow. 
In the case of carbon ions, the larger amount of secondary neutron produced during the patient treatment seems to require other background rejection methods in order to lead to an advantageous signal-over-background ratio. Note also that online filtering strategies  may be used to improve the quality of the date: already in figure \ref{fig:coincidences}, one can notice that the amount of reconstructed events is about half that of true gamma, which means that partial absorption in the absorber leads to events which cannot be reconstructed via the line-cone method. More refined pre-analysis could be used and have been proposed in \cite{Draeger:2017aa}.

The camera precision has been estimated starting from a reference prompt gamma emission profile obtained at high primary particle statistics ($10^{10}$), with the random extraction of 1000 data subsets per intensity and applying a minimization robust algorithm to define the shift of the identified profile fall-off with respect to the reference one.
The camera precision in the falloff identification rapidly increases for increasing primary particle statistics. A sufficiently good precision is achieved on a spot basis for proton beams, where the precision is about 2~mm with a MLEM reconstruction: a qualitative monitoring of each spot seems then possible. In order to achieve millimeter or sub-millimeter precision, some spot grouping methods must be considered. As a general result, the LM-MLEM iterative algorithm, which is now the standard for this kind of image reconstruction,  guarantees a better fall-off identification precision over the whole explored intensity range. However, MLEM does not exploit the additional information provided by the knowledge of the beam position in the transverse plane. In future studies, such a priori information should be considered in order to improve the reconstruction rapidity and image quality. At present, given the long calculation time required by the iterative algorithm, the line-cone reconstruction method can still be an option for on-line treatment check, when safety limit can be fixed in order to exclude severe deviation from the treatment planning and an interruption of the dose delivery in real time can be foreseen. Moreover, the TOF information can be included in the line-cone reconstruction method in order to constrain the emission on a single point, and improve its accuracy.

The Compton detection principle has already proven its potential in detecting prompt gammas for ion beam therapy monitoring purpose, and the CLaRyS Compton camera prototype shows promising results for this application. The detector is now at the final development stage, its components are being tested on beam in clinical facilities and a first beam test with a complete system is foreseen for the next year. New simulation studies are to ba carried out to benchmark the Compton camera device to other detection systems, like PET machines or collimated detectors, already used or tested in clinics for ion beam therapy monitoring.            


