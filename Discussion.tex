\section{Discussion}



Therefore, the Compton camera position optimization with respect to the expected Bragg peak position appears mandatory for an efficient fall-off reconstruction.\newline

A further study showed how this effect is mainly due to the cut applied on the scatterer detector. Indeed, for low energies, below 2~MeV, the increased efficiency in the central section of the camera active surface shown in figure~\ref{fig::efficiency_study}(a) is linked to the increased relative number of photons approaching the camera with small angles. These photons are more likely undergoing Compton scattering with a reduced energy deposition, which is recorded by an ideal detector and rejected by the energy cut. The effect is all the more important as the primary gamma energy is limited, creating the peculiar energy dependence of the efficiency reduction emerged in the results in figure~\ref{fig::efficiency_study}(b).
 
Furthermore, considering the rate of single interactions per detector, the results obtained by a preliminary analysis showed, as an example, a single rate of about 300~MHz on the absorber and 20~MHz on the first scatterer planes. These rates are not compatible with the detection rate capabilities of the detector read-out chains. As a result, it appears not possible to perform a valuable treatment monitoring with the CLaRyS Compton camera at a clinical beam intensity.\newline

In the case of carbon ions, the larger amount of secondary neutron produced during the patient treatment seems to require other background rejection methods in order to lead to an advantageous signal-over-background ratio.\\

If we consider the important increase in the precision of the falloff identification given by the increased statistics, the spot grouping method seems to be promising for monitoring purpose when high accuracy is required, probably obtained in the post-treatment due to the long required reconstruction time. A sufficiently good precision is achieved on a spot basis, where the precision is about 2~mm with a MLEM reconstruction: a qualitative monitoring of each spot seems then possible. Given the long calculation time required by the MLEM algorithm, the line-cone reconstruction method, despite the reduced precision, can still be an option for on-line treatment check, when safety limit can be fixed in order to exclude severe deviation from the treatment planning and an interruption of the dose delivery in real time can be foreseen.          


