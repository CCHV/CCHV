\section{Introduction}\label{section::Intro}
Ion beam therapy is a cancer treatment technique which is rapidly gaining importance in the global tumor therapy panorama. In addition to the already operational 70 clinical facilities, for more than 175000 patients already treated by the end of 2017, several new centers have been designed and approved for construction worldwide~\cite{PTCOG_stats}. The favorable feature of this treatment technique is connected to the peculiar energy deposition profile of charged particles as a function of depth in matter. As first observed by Bragg~\cite{Bragg_main}, the depth-dose profile of charged particles shows a maximum close to the end of their range in matter; in addition to this, a strong enhancement of the relative biological effectiveness (RBE - capability of inducing DNA damages at fixed dose) is observed for ions heavier than protons in the region of the Bragg peak~\cite{RBE_Elsasser, RBE_Weyrather}, which enables a reduction of the biological dose in healthy tissues.

The Bragg peak position corresponds to the maximum of the dose deposited in the patient and must be tuned to cover the target volume and, at the same time, spare the surrounding healthy tissues. The tumor volume is defined via CT scan in the standard clinical routine, and the treatment is planned via treatment planning software. In human body, the exploitation of the high ballistic precision of ion beams is strongly limited by treatment planning and delivery uncertainties, like uncertainties in the material composition determination, CT units conversion to ion stopping power, patient mis-positioning, organ motion or morphological changes between treatment fractions (a standard treatment is divided into several fractions over several weeks, according to the tumor and patient characteristics). These uncertainties force the clinicians to fix relatively large safety margins around the planned treatment volume, up to 3.5\% + 3 mm~\cite{Paganetti:2012aa}. The research and clinics community agrees on the fact that ion-range verification is one of the conditions (at present the main condition) required for a broader usage of ion beam therapy and for its further development. With the goal of fully exploiting the ion beam therapy dosimetric potential, the monitoring should be in real-time and ideally in 3 dimensions, in order to be able to interrupt the treatment in case of severe issues, corresponding to important differences between the planned and delivered dose to the target volume or to surrounding organs, in particular in case of proximity to organs-at-risk~(OAR).

Several techniques have been considered worldwide for twenty years. Most of the studied techniques rely on the detection of secondary radiations generated during the slowing down process of incident ions, in particular during nuclear reactions. Among theses secondary radiations, positron emitters have been deeply studied in order to exploit positron emission tomography (PET) machines for treatment monitoring. PET techniques are based on the detection of the two back-to-back 511~keV photons produced by the annihilation of positrons (created by the emitter fragments of nuclear reactions) with patient electrons, resulting in a delayed radiation which should be detected with time coincidences, allowing for an intrinsic background reduction. Nevertheless, the monitoring with positron emitters secondary signal must deal with a limited count rate compared to medical imaging PET, with the lifetime of emitters providing a delayed information that implies the signal integration over a whole treatment fraction (not a single spot or group of spots), with physiological washout effects depending on to the emitters lifetime.

Even if the only available and functional range monitoring system in a clinical center is based on this technique~\cite{ENGHARDT2004}, several clinical experience with commercial or adapted PET system already shown intrinsic limitations mainly connected to the ring geometry (not directly applicable to the treatment monitoring due to the presence of the beam) or in general to geometrical constraints limiting the field of view and the resulting system global efficiency and spatial accuracy (the limited detection angle generates artifacts in the final image)~\cite{PARODI2016}. The research is ongoing and new results are expected for the next years thanks to the introductions of new systems with adapted geometries, to the improvements in acquisition and reconstruction techniques and to the clinical introduction of time-of-flight systems, intrinsically able to improve the detector spatial resolution via interaction time information, and depth-of-interaction reconstruction, which will allow for a more precise spatial reconstruction for reduced angular artifacts effects.

In addition to positron annihilation products, a different kind of photons is emitted during ion irradiation. They emerge from the relaxation of excited nuclei in a wide energy range, between some hundreds of keV till about 8-10 MeV. After the first idea proposal published in 2003~\cite{PG_first}, these secondary products of particle treatment have been deeply investigated and the correlation of this gamma radiation to the ion depth-dose profile has been confirmed by several research groups, starting from~\cite{Min_PG} for protons and~\cite{Testa_PG} for heavier ions (mainly carbon). The so-called prompt gamma-rays (PG) have the advantage to be emitted almost instantaneously after the beam interaction in the tissue, making them more adapted than PET 511~keV gammas for real-time monitoring. Consequently, different techniques have been proposed to exploit this signal for treatment monitoring purpose, with the related detection systems. Some methods are based on PG timing~\cite{Golnik:2014aa, Krimmer_PGPI} or energy~\cite{Verburg:2014aa} information and rely on non-collimated systems; more complex detection apparatus can achieve an actual PG imaging, by exploiting mechanical or electronic collimation (i.e.~with collimated gamma camera or Compton camera, respectively) for the photon selection (see e.g.~\cite{Min_PG, Bom_collimated, Smeets:2012aa, Roellinghoff_2014, Priegnitz:2015aa, Frandes_2010, LLOSA2012105, KORMOLL2011114, MCCLESKEY2015163, Matsuoka:2014qna, Peterson:2010aa, Solevi:2016aa, ALDAWOOD2017190}). For a review on PG monitoring, see~\cite{krimmer:hal-01585334}

Originally designed for astrophysics applications, the potential of Compton cameras for medical imaging has been soon recognized~\cite{TODD:1974aa} and then directly translated to the ion beam therapy monitoring domain. Such a gamma detection system is generally  composed of two sections: a scatterer and an absorber. The scatterer is dedicated to the gamma Compton scattering, and should be so adapted in order to optimize the Compton scattering probability in the prompt gamma energy range, while reducing the so called Doppler broadening effect due to electron bounding and motion~\cite{Doppler}; this leads, in most of the cases, to the choice of a light material (low $Z$), segmented in several subsections. However, when efficiency has to be privileged, heavier materials may be used~(\cite{Solevi:2016aa, ALDAWOOD2017190, 0031-9155-60-18-7085}). The absorber is devoted to the final absorption of the Compton scattered photons via photoelectric effect; it is often composed of segmented high-$Z$ scintillating materials. Slightly different Compton camera configuration can also achieve Compton electron tracking in the scattering detector~\cite{Frandes_2010, Yoshihara_ETCC}, which results in additional information for the further reconstruction algorithm.
 The collected interaction positions and energy depositions in the two detector sections are used to limit the emission point on the surface of a cone (or to a segment in the cone if the Compton electron track information is retrieved), via Compton kinematics reported in equation~\ref{Compton_equation} (for an electron initially at rest):
\begin{equation}
\cos\theta\,=\,1-\frac{m_{e}c^{2}E_{1}}{E_{2}(E_{1}+E_{2})},
\label{Compton_equation}
\end{equation} 
where \(m_{e}c^{2} = 511\)~keV, \(E_{1}\) and \(E_{2}\) are the energies, respectively, deposited in the scatterer and the absorber. 
Once the cones are calculated, analytic or iterative algorithms are used to create the image of the prompt gamma emission distribution, with intrinsic 3 dimensional capability~\cite{McKisson3D, Kuchment:2016uiw}. 


Several sources of uncertainty and signal background are connected to this detection method. Considering a gamma interacting twice in the detector, i.e.~once in the scatterer and once in the absorber, a complete photon energy absorption is needed since the initial photon energy (\(E_{1}\)) is not known a priori and the reported formula assumes valid the relation in equation~\ref{energy_equation}:
 \begin{equation}
E_{0} = E_{1}+E_{2}.
\label{energy_equation}
\end{equation} 
An underestimation of the total initial energy (caused by a photon non-complete absorption in the absorber section or by the Compton electron escape from the scatterer section), leads to an underestimation of the Compton angle, so to a Compton cone reconstruction incertitude. For a three-interaction operating mode (making use of double scattered photons), the initial photon energy can be calculated analytically so that a complete absorption is not mandatory. In addition to this, the Compton kinematics formula does not take into account the initial bounding configuration of the Compton scattering electron, which creates a blur in the Compton angle reconstruction, resulting in the already cited Doppler broadening effect~\cite{Doppler}. Furthermore, the time structure of the incoming particles plays an important role due to the detection principle, which is based on time coincidences between the two detector sections. The final image accuracy then suffers from false coincidences, generated by two prompt gammas interacting in the same time window or by the contamination of different kind of secondaries, mainly neutrons and protons. It is clear that excellent detector time resolution can limit the amount of random coincidences, as well as different background rejection methods can be applied to select real coincidences~\cite{Draeger:2017aa}. Energy selections can be applied to the collected coincidences~\cite{Polf:2009aa, Hilaire:2016aa}, or the homogeneous neutron background can be reduced via time-of-flight information~\cite{Testa:2010aa}.

A French collaboration of 4 research institutions (Institut de Physique Nucleaire and Centre de Recherche en Acquisition et Traitement de l'Image pour la Sant\'e in Lyon, Laboratoire de Physique Subatomique et Cosmologie in Grenoble, Centre de Physique des Particules in Marseille) is focused on the design and development of gamma detectors for online ion beam therapy monitoring. The project includes the development and clinical test of a Compton camera prototype based on semiconductor and scintillator detectors~\cite{krimmer:hal-01101334}; the feasibility of its clinical application is studied in this work with Monte Carlo simulations.
 
In a previous study by Ortega and colleagues~\cite{Ortega:2015aa} a detailed analysis of the noise sources for Compton imaging in proton therapy monitoring is presented, and the clinical application of this method for detecting range shifts is tested. The simulation study, performed with a simple Compton camera prototype composed by multiple LaBr$_3$ layers with a ideal mono-energetic proton beam impinging on a PMMA phantom, showed the relative expected rate of prompt gammas and neutrons, and the resulting rate of random coincidences ranging from 19 \% to more than 60 \% depending on the beam energy and the coincidence time window. This amount of fake events leads to complex reconstruction scenarios, where the identification of a 3~mm range shift is not clear for all cases.
    
Starting from these results, the ClaRyS camera performances are here studied as a function of the gamma energy in the prompt gamma energy range, and the possible application of this detector as depth-dose profile monitor during ion beam therapy clinical treatment is analyzed by investigating the effect of the beam time structure on the rate of true and random coincidences, as well as on the background level. After a preliminary study with point-like gamma sources irradiation focused on detector efficiency measurements as a function of the source position and gamma energy, clinical proton and carbon beams impinging on an homogeneous PMMA phantom are simulated to reproduce treatment conditions and analyze the prompt gamma detection resulting scenario. A beam intensity reduction with respect to treatment intensities is studied for monitoring purpose, with the aim of increasing the ratio between true and random coincidence events: the maximum beam rate tolerance is here estimated. Two kinds of reconstruction algorithms, a line-cone analytic method and a MLEM iterative one, are applied to the collected data in order to compare the imaging results. Finally, the Compton camera precision in the identification of the dose profile fall-off is reported.   
