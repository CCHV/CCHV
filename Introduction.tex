\section{Introduction}

The exploitation of the high ballistic precision of hadrontherapy is currently limited by treatment uncertainties such like patient mispositioning, organ motion or morphological  changes between treatment fractions. These uncertainties lead to the definition of relatively large safety margins around the planning treatment volume. Moreover the optimal treatment fields with a better sparing of healthy tissues are often discarded to avoid the irradiation of tumor with an organ at risk located downstream. There is therefore a global consensus in the community that ion-range verification is one of the conditions required for a broader usage of hadrontherapy. 

Several techniques have been considered worldwide for twenty years. Most of them rely on the detection of secondary radiations generated during nuclear reactions undergone by a fraction of incident ions. Among theses secondary radiations, prompt gamma-rays (PG) has the advantage to be emitted almost instantaneously which could allow for online treatment verification. In addition to this temporal feature, most of PG have well-defined energies corresponding to discrete transitions of incident ions (for ions heavier than protons) or target nuclei (mainly carbon and oxygen). Three observables can be therefore considered: emission point, PG energy and time-of-flight (TOF). Various modalities have been investigated taking benefit from one or several observables: PG imaging with or without collimation (i.e. collimated gamma camera or Compton camera, respectively)
